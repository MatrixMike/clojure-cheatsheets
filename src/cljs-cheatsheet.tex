\documentclass[footexclude,twocolumn,DIV25,fontsize=10pt]{scrreprt} 

% Author: Fogus (based on Steve Tayon's Clojure and David Liebke's Incanter cheat sheets)
% Comments, errors, suggestions: fogus -at- clojure -dot- com

% License
% Eclipse Public License v1.0
% http://opensource.org/licenses/eclipse-1.0.php

% Packages
\usepackage[utf8]{inputenc}
\usepackage[T1]{fontenc}
\usepackage{textcomp}
\usepackage[english]{babel}
\usepackage{tabularx}

\usepackage[table]{xcolor}

% Set column space
\setlength{\columnsep}{0.25em}

% Define colours
\definecolorset{hsb}{}{}{red,0,.4,0.95;orange,.1,.4,0.95;green,.25,.4,0.95;yellow,.15,.4,0.95}

\definecolorset{hsb}{}{}{blue,.55,.4,0.95;purple,.7,.4,0.95;pink,.8,.4,0.95;blue2,.58,.4,0.95}

\definecolorset{hsb}{}{}
{magenta,.9,.4,0.95;green2,.29,.4,0.95}

% Redefine sections
\makeatletter
\renewcommand{\section}{\@startsection{section}{1}{0mm}
  {-1.7ex}{0.7ex}{\normalfont\large\bfseries}}
\renewcommand{\subsection}{\@startsection{subsection}{2}{0mm}
  {-1.7ex}{0.5ex}{\normalfont\normalsize\bfseries}}
\makeatother

% No section numbers
\setcounter{secnumdepth}{0}

% No indentation
\setlength{\parindent}{0em}

% No header and footer
\pagestyle{empty}


% A few shortcuts
\newcommand{\cmd}[1] {\texttt{\textbf{{#1}}}}
\newcommand{\cmdline}[1] {
  \begin{tabularx}{\hsize}{X}
    \texttt{\textbf{{#1}}}
    \end{tabularx}
}

\newcommand{\colouredbox}[2] {
  \colorbox{#1!40}{
    \begin{minipage}{0.95\linewidth}
      {
        \rowcolors[]{1}{#1!20}{#1!10}
        #2
        }
      \end{minipage}
    }
}

\begin{document}

\centerline{\Large{\textbf{ClojureScript Cheat Sheet}}}
\centerline{{\large{\textbf{\textit{http://github.com/clojure/clojurescript}}}}}

\colouredbox{green}{
\section{Documentation}
\textmd{\textrm{http://github.com/clojure/clojurescript/wiki}}
}

\colouredbox{red}{
\section{Rich Data Literals}
\begin{tabularx}{\hsize}{lX}
Maps: & \cmd{\{:key1 :val1, :key2 :val2\}} \\
Vectors: & \cmd{[1 2 3 4 :a :b :c 1 2]} \\
Sets: & \cmd{\#\{:a :b :c 1 2 3\}} \\
Truth and nullity: & \cmd{true, false, nil} \\
Keywords: & \cmd{:kw, :a-2, :prefix/kw, ::pi} \\
Symbols: & \cmd{sym, sym-2, prefix/sym} \\
Characters: & \cmd{\textbackslash a, \textbackslash u1123, \textbackslash space} \\
Integers: & same as in JavaScript \\
Floats: & same as in JavaScript \\
Strings: & same as in JavaScript \\
\end{tabularx}

}


\colouredbox{blue}{
\section{It's (mostly) Clojure}

\subsection{Frequently Used Functions}
\begin{tabularx}{\hsize}{lX}
Math:& \cmd{+ - * / quot rem mod inc dec max min}\\
Comparison:& \cmd{= == not= < > <= >=}\\
Tests:& \cmd{nil? identical? zero? pos? neg? even? odd? true? false? nil?} \\
Keywords:& \cmd{keyword keyword?} \\
Symbols:& \cmd{symbol symbol? gensym} \\
Macros:& \cmd{if if-let cond and or -> ->> doto when when-let ..}\\
Data Processing:& \cmd{map reduce filter partition split-at split-with} \\
Data Create:& \cmd{vector vec hash-map set list list*} \\
Data Examination:& \cmd{first rest count get nth get get-in contains? find keys vals} \\
Data Manipulation:& \cmd{seq into conj cons assoc assoc-in dissoc zipmap merge merge-with select-keys update-in}\\
Arrays:& \cmd{into-array to-array aget aset}\\
\end{tabularx}


\subsection{Macros}
\begin{tabularx}{\hsize}{lX}
Organization:& in a \cmd{cljs-macros} directory in the source home\\
Importing:& \cmd{(ns foo (:require-macros [my.macros :as macs]))}\\
Defining:& \cmd{defmacro}\\
Implementation:& Must be written in Clojure \\
Emission:& Must emit ClojureScript \\
\end{tabularx}

\subsection{More information}
\begin{tabularx}{\hsize}{lX}
\centerline{{\large{\textbf{\textit{http://clojuredocs.org}}}}}
\end{tabularx}

}


\colouredbox{yellow}{
\section{JavaScript Interop}
\begin{tabularx}{\hsize}{lX}
Foo: & \cmd{matrix diag identity-matrix symmetric-matrix bind-columns bind-rows to-matrix} \\
Bar: & \cmd{dim ncol nrow rank condition matrix?} \\
Baz: & \cmd{sel diag group-by} \\
Quux: & \cmd{plus minus mult div exp log log10 log2 pow sin cos tan asin acos atan abs} \\
\end{tabularx}

}


\colouredbox{blue2}{
\section{Compiler}
\begin{tabularx}{\hsize}{lX}
Foo: & \cmdline{cljsc --blah} \\
Bar: & \cmdline{cljsc --blah} \\
\end{tabularx}

}


\colouredbox{magenta}{
\section{Namespaces}
\begin{tabularx}{\hsize}{lX}
Foo: & \cmd{dim ncol nrow dataset?} \\
Bar: & \cmd{sel group-by} \\
Baz: & \cmd{to-matrix} \\
Quux: & \cmd{view} \\
\end{tabularx}

}


\colouredbox{orange}{
\section{Creating Abstractions}

\subsection{Protocols}
\begin{tabularx}{\hsize}{lX}
Foo:& \cmd{bar-chart line-chart}\\
Bar:& \cmd{add-category}\\
Baz:& \cmd{add-category}\\
Quux:& \cmd{add-category}\\
\end{tabularx}

\subsection{Records}
\begin{tabularx}{\hsize}{lX}
Foo:& \cmd{xy-plot scatter-plot}\\
Bar:& \cmd{add-lines add-points add-function}\\
Baz:& \cmd{add-points add-polygon, add-text} \\
Quux:& \cmd{add-points add-polygon, add-text} \\
Fizz:& \cmd{add-points add-polygon, add-text} \\
\end{tabularx}

\subsection{Types}
\begin{tabularx}{\hsize}{lX}
Foo:& \cmd{bar-chart line-chart}\\
Bar:& \cmd{add-category}\\
Baz:& \cmd{add-category}\\
Quux:& \cmd{add-category}\\
Fizz:& \cmd{add-category}\\
\end{tabularx}

\subsection{Reify}
\begin{tabularx}{\hsize}{lX}
Foo:& \cmd{bar-chart line-chart}\\
Bar:& \cmd{add-category}\\
\end{tabularx}

\subsection{Multimethods}
\begin{tabularx}{\hsize}{lX}
Foo:& \cmd{bar-chart line-chart}\\
Bar:& \cmd{add-category}\\
\end{tabularx}

}


\colouredbox{pink}{
\section{ClojureScript Core Utils}
\begin{tabularx}{\hsize}{lX}
Strings:& \cmd{bar-chart line-chart}\\
Sets:& \cmd{add-category}\\
Zipper:& \cmd{add-category}\\
Browser Utils:& \cmd{add-category}\\
\end{tabularx}

}


\colouredbox{green}{
\section{Other Useful Libraries}
\begin{tabularx}{\hsize}{lX}
Foo: & http://github.com/levand/domina \\
Foo: & http://github.com/levand/domina \\
Foo: & http://github.com/levand/domina \\
Foo: & http://github.com/levand/domina \\
Foo: & http://github.com/levand/domina \\
\end{tabularx}

}


\begin{flushright}
\footnotesize
\rule{0.7\linewidth}{0.25pt}
\verb!$Revision: 0.1, $Date: Feb 08, 2012!\\
\verb!Fogus (fogus -at- clojure -dot- com)!
\end{flushright}
\end{document}
